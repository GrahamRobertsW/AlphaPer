\documentclass[12pt]{article}
\usepackage[pages=some,scale=1,angle=0,opacity=0.4]{background}
\usepackage{graphicx}
\usepackage{mathtools}
\usepackage{amsmath}
\usepackage[paperwidth=11in,paperheight=11in,margin=.75in]{geometry}
\fontsize{.375in}{.25in}
\begin{document}
\BgThispage
\pagestyle{empty}
\backgroundsetup{contents=\includegraphics{pmdistbg}}
\centerline{\Huge{ABSTRACT}}
\huge{Alpha Persei is a young stellar open cluster in our Galaxy.  Stellar open clusters are groups of stars that formed at the same time from a single cloud in the interstellar medium.  Cluster’s well defined ages allow astronomers to calibrate stellar evolution models with measurements of constituent members, and track the history of star formation in the Milky Way.  Alpha Persei is relatively young, on the scale of around 100 million years old (Shiekhi et al. 2016).  Alpha Per provides a key laboratory for studying the properties, such as mass, radii, temperature, etc. of young stars.  We have yet to identify enough members of the cluster to more accurately estimate its age.  This is why a membership survey is important.  Since stellar clusters form close together the members interact gravitationally resulting in a mass of stars that are very close together.  More interestingly they all move at nearly the same velocity through the Galaxy. Alpha Per’s proximity is such that the cluster’s velocity through space produces a large proper motion, or apparent angular momentum in the plane of the sky, relative to other stars in our Galaxy.  We model the cluster population of stars using each star’s position in the night sky as well as proper motion. Analyzing these distributions we reduce a database down to a manageable number of stars that are  probably also members of the cluster, down to under 100,000 from 4 million.  This membership catalogue will allow us to complete a more precise study of the cluster’s age.} 
\end{document}
